\documentclass[svgnames]{article}
\usepackage[utf8]{inputenc}
\usepackage{amsmath}
\usepackage{amssymb}
\usepackage{mathrsfs}
\usepackage{mathtools}
\newtheorem{mydef}{Given}
\newtheorem{mytheorem}{Theorem}
\usepackage{enumitem}
\usepackage{venndiagram}
\usepackage{smartdiagram}
\usepackage{caption}
\usepackage{subcaption}
%\usepackage[framed,numbered,autolinebreaks,useliterate]{mcode}
\usepackage{pgfplots}
%\usepackage{tkiz}
\usepackage{listings}
\definecolor{dkgreen}{rgb}{0,0.6,0}
\definecolor{gray}{rgb}{0.5,0.5,0.5}
\definecolor{mauve}{rgb}{0.58,0,0.82}
\usepackage{float}

\lstset{ %
  language=R,                     % the language of the code
  basicstyle=\footnotesize,       % the size of the fonts that are used for the code
  numbers=left,                   % where to put the line-numbers
  numberstyle=\tiny\color{gray},  % the style that is used for the line-numbers
  stepnumber=1,                   % the step between two line-numbers. If it's 1, each line
                                  % will be numbered
  numbersep=5pt,                  % how far the line-numbers are from the code
  backgroundcolor=\color{white},  % choose the background color. You must add \usepackage{color}
  showspaces=false,               % show spaces adding particular underscores
  showstringspaces=false,         % underline spaces within strings
  showtabs=false,                 % show tabs within strings adding particular underscores
  frame=single,                   % adds a frame around the code
  rulecolor=\color{black},        % if not set, the frame-color may be changed on line-breaks within not-black text (e.g. commens (green here))
  tabsize=2,                      % sets default tabsize to 2 spaces
  captionpos=b,                   % sets the caption-position to bottom
  breaklines=true,                % sets automatic line breaking
  breakatwhitespace=false,        % sets if automatic breaks should only happen at whitespace
  title=\lstname,                 % show the filename of files included with \lstinputlisting;
                                  % also try caption instead of title
  keywordstyle=\color{blue},      % keyword style
  commentstyle=\color{dkgreen},   % comment style
  stringstyle=\color{mauve},      % string literal style
  escapeinside={\%*}{*)},         % if you want to add a comment within your code
  morekeywords={*,...}            % if you want to add more keywords to the set
} 


%\renewcommand{\theenumi}{\Alph{enumi}}
\newenvironment{amatrix}[1]{%
  \left(\begin{array}{@{}*{#1}{c}|c@{}}
}{%
  \end{array}\right)
}

\newenvironment{tolerant}[1]{%
  \par\tolerance=#1\relax
}{%
  \par
}


\pgfmathdeclarefunction{gauss}{2}{%
  \pgfmathparse{1/(#2*sqrt(2*pi))*exp(-((x-#1)^2)/(2*#2^2))}%
}


\title{Statistical methods: Homework 10}
\author{Cameron McIntyre}
\date{\today}

\begin{document}

\maketitle

\section{9.2.6}

Ring Lardner was one of this country’s most popular writers during the 1920s and 1930s. He was also a chronic alcoholic who died prematurely at the age of forty-eight. The following table lists the life spans of some of Lardner’s contemporaries (39). Those in the sample on the left were all problem drinkers; they died, on the average, at age sixty-five. The twelve (sober) writers on the right tended to live a full ten years longer. Can it be argued that an increase of that magnitude is statistically significant? Test an appropriate null hypothesis against a one-sided H1. Use the 0.05 level of significance. (Note: The pooled sample standard deviation for these two samples is 13.9.)

\begin{table}[H]
\centering
 \begin{tabular}{l c c l c} 
 \hline
\multicolumn{2}{c}{\bf Authors Noted for}  & & \multicolumn{2}{c}{\bf Authors Not Noted for} \\
\multicolumn{2}{c}{\bf Alcohol Abuse}  & &\multicolumn{2}{c}{\bf Alcohol Abuse}\\
\cline{1-2} \cline{4-5} 
\bf  & \bf Age at && \bf  & \bf Age at \\
\bf  Name & \bf Death && \bf  Name& \bf Death \\
\hline
Ring Lardner & 48  && Carl Van Doren  & 65 \\
Sinclair Lewis &  66  && Ezra Pound &  87  \\
Raymond Chandler &  71  && Randolph Bourne &  32  \\
Eugene O'Neill  & 65 &&  Van Wyck Brooks  & 77\\
Robert Blenchley & 56 && Samuel Elliot Morrison & 89\\
J.P. Marquand & 67 && John Crowe Ransom & 86\\
Dashiell Hammett & 67 && T.S. Eliot & 77 \\
e.e. cummings & 70 && Conrad Aiken & 84\\
Edmund Wilson & 77 && Ben Ames Williams \\
Average: & 65.2  && Henry Miller & 88 \\
&   && Archibald MacLeish & 90 \\
&   && James Thurbur & 67 \\
&   && Average & 75.5 \\
 \hline
 \end{tabular}
\end{table}

\subsection*{Answer:}
\textbf{H$_0$}: $\mu_{alc} = \mu_{N-alc}$
\newline
\textbf{H$_1$}:  $\mu_{alc} < \mu_{n-alc}$

Our test statistic:

$$\frac{\mu_{alc} - \mu_{n - alc}}{s_p\sqrt{\frac{1}{n}+\frac{1}{m}}}=\frac{65.22-75.5}{13.9\sqrt{\frac{1}{9}+\frac{1}{12}}}=-1.68$$

Our critical value is $t_{.05,19}=-1.729$

Since our test statistic is not less that -1.72, we do not reject $H_0$.


\section{9.2.12}

Suppose that $H_0: \mu X = \mu Y$ is being tested against $H_1: \mu X \neq \mu Y$ ,where $\sigma^2_X$ and $\sigma^2_Y$ are known to be 17.6 and 22.9, respectively. If $n = 10$, $m = 20$, $x = 81.6$, and $y = 79.9$, what P-value would be associated with the observed $Z$ ratio?

\subsection*{Answer:}
\textbf{H$_0: \mu X = \mu Y$}
\newline
\textbf{H$_1: \mu X \neq \mu Y$}
\newline
Our test statistic is calculated:
$$\frac{ \bar{X} - \bar{Y}}{\sqrt{\frac{s_x^2}{N} + \frac{s_y^2}{M}}} = \frac{81.6 - 79.9}{\frac{17.6}{10}+\frac{22.9}{20}} = 0.997$$

The P value we use R:

\begin{lstlisting}
> pnorm(.997, lower.tail=FALSE)
[1] 0.1593823
> #two tailed
> 2*pnorm(.997, lower.tail=FALSE)
[1] 0.3187645
\end{lstlisting}

Therefore our p value is 0.3187645. 

\section{9.2.20}

Two popular forms of mortgage are the thirty-year fixed-rate mortgage, where the borrower has thirty years to repay the loan at a constant rate, and the adjustable rate mortgage (ARM), one version of which is for five years with the possibility of yearly changes in the interest rate. Since the ARM offers less certainty, its rates are usually lower than those of fixed-rate mortgages. Test this hypothesis at the $\alpha = 0.01$ level using the following sample of mortgage offerings for a loan of \$250,000. Do not assume the variances are equal.

\subsection*{Answer:}
30 Year Fixed:
$\bar{x}_{fixed}=3.673$
$sd_{fixed} = .153$
n=10
ARM:
$\bar{x}_{arm}=3.338$
$sd_{arm} = .216$
n=10
\newline
\textbf{H$_0: \bar{x}_{fixed} = \bar{x}_{arm} $}
\newline
\textbf{H$_1:\bar{x}_{fixed} \neq \bar{x}_{arm} $}
\newline

We use Welch's t-test to calculate our test statistic.
$$\frac{\bar{x}_{fixed}- \bar{x}_{arm} }{\sqrt{\frac{sd_{fixed}}{n}+\frac{sd_{arm}}{m}}}=\frac{3.673 - 3.338}{\sqrt{\frac{.152^2}{10}+\frac{.216^2}{10}}}=4$$

Now we calculate the degrees of freedom.
$$df=\frac{\frac{s_1^1}{n}+\frac{s_2^2}{m}}{\frac{\frac{s_{fixed}^2}{n}^2}{n-1}+\frac{\frac{s_{arm}^2}{n}^2}{n-1}} = \frac{\frac{.153^2}{10}+\frac{.216^2}{10}}{\frac{\frac{.152^2}{10}}{9}+\frac{\frac{.216^2}{10}^2}{9}}=16$$

Pvalue= .000516, We reject the null hypothesis. Therefore the average rates are not the same between fixed and floating mortgages. 

\section{9.3.4}

In a study designed to investigate the effects of a strong magnetic field on the early development of mice (7), ten cages, each containing three 30-day-old albino fe-male mice, were subjected for a period of 12 days to a magnetic field having an average strength of 80 Oe/cm. Thirty other mice, housed in ten similar cages, were not put in the magnetic field and served as controls. Listed in the table are the weight gains, in grams, for each of the twenty sets of mice/ 

\begin{table}[H]
\centering
 \begin{tabular}{l c c l c} 
 \hline
\multicolumn{2}{c}{\bf In Magnetic Field}  & & \multicolumn{2}{c}{\bf Not In Magnetic Field} \\
\cline{1-2} \cline{4-5} 
\bf  Cage & \bf  Weight Gain (g) && \bf  Cage & \bf  Weight Gain (g)  \\
\hline
1 & 22.8  && 11  & 23.5 \\
2 &  10.2  && 12 &  31.0  \\
3 &  20.8  && 13 &  19.5  \\
4  & 27.0 &&  14  & 26.2\\
5 & 19.2 && 15 & 26.5\\
6 & 9.0 && 16 & 25.2\\
7 & 14.2 && 17 & 24.5 \\
8 & 19.8 && 18 & 23.8\\
9 & 14.5 && 19 & 27.8\\
10 & 14.8  && 20 & 22.0 \\
 \hline
 \end{tabular}
\end{table}


Test whether the variances of the two sets of weight gains are significantly different. $Let \alpha = 0.05$. For the mice in the magnetic field, $s_X = 5.67$; for the other mice, $s_Y = 3.18$.

\subsection*{Answer:}
\textbf{H$_0: \sigma_x = \sigma_{y} $}
\newline
\textbf{H$_1: \sigma_x \neq \sigma_{y} $}
\newline

Remember, $\frac{s_x}{s_y} \sim F_{m-1,n-1}$.\newline
Therefore, our test statistic is:

$$\frac{s_y^2}{s_x^2}=\frac{3.18^2}{5.67^2}=.314$$

We need to get our critical values:
$m-1=9$\newline
$ n-1=9$\newline
$F_{.025,9,9}=.248$\newline
$F_{.975,9,9}=4.03$\newline

Since .314 is in the interval (.248, 4.03) we do not reject the null hypothesis. 


\section{9.4.4}

If flying saucers are a genuine phenomenon, it would follow that the nature of sightings (that is, their physical characteristics) would be similar in different parts of the world. A prominent UFO investigator compiled a listing of ninety-one sightings reported in Spain and eleven hundred seventeen reported elsewhere. Among the information recorded was whether the saucer was on the ground or hovering. His data are summarized in the following table (95).Let pS and pNS denote the true probabilities of "Saucer on ground" in Spain and not in Spain, respectively. Test $H_0: p_S = p_{NS}$ against a two-sided $H_1$.Let $\alpha = 0.01$.


\begin{table}[H]
\centering
 \begin{tabular}{l c c} 
 \hline
\bf   & \bf  In Spain & \bf  Not in Spain \\
\hline
Saucer on ground & 53  & 705 \\
Saucer hovering &  38  & 412  \\
 \hline
 \end{tabular}
\end{table}


Test whether the variances of the two sets of weight gains are significantly different. $Let \alpha = 0.05$. For the mice in the magnetic field, $s_X = 5.67$; for the other mice, $s_Y = 3.18$.

\subsection*{Answer:}
\textbf{H$_0: p_{S} = p_{NS} $}
\newline
\textbf{H$_1: p_{S} \neq p_{NS}$}
\newline

Our estimates\newline
$p_{S}  = \frac{53}{91}=.582$ \newline $ p_{S} \frac{705}{1117}=.6311$
\newline
Our pooled estimates:
$\hat{p} = \frac{758}{1208}=.627$
\newline
Our critical value $z_{\frac{\alpha}{2}}=z_{.005}=-2.58$

Now we calculate our test statistic,
$$\frac{.582 - .6311}{\sqrt{\frac{.627(1-.627)}{91} + \frac{.627(1-.627)}{1208}}}=\frac{-.049}{.0527}=-.93$$
.93 is in the interval (-2.58, 2.58). Therefore we do not reject $H_0$.


\section{9.5.2}

Male fiddler crabs solicit attention from the oppo-site sex by standing in front of their burrows and waving their claws at the females who walk by. If a female likes what she sees, she pays the male a brief visit in his bur-row. If everything goes well and the crustacean chemistry clicks, she will stay a little longer and mate. In what may be a ploy to lessen the risk of spending the night alone, some of the males build elaborate mud domes over their burrows. Do the following data (229) suggest that a male’s time spent waving to females is influenced by whether his burrow has a dome? Answer the question by constructing and interpreting a 95\% confidence interval for $\mu_X - \mu_Y$. Use the value $s_p = 11.2$.


\begin{table}[H]
\centering
 \begin{tabular}{c c} 
 \hline
\multicolumn{2}{c}{\bf \% of Time Spend Waving to Females}\\
\hline 
\bf  Males with Domes, $x_i$ & \bf  Males without Domes, $y_i$  \\
\hline
100.0 & 76.4  \\
58.6 &  84.2  \\
93.5 & 96.5  \\
83.6 &  88.8  \\
84.1 & 85.3  \\
 &  79.1  \\
 & 83.6  \\
 \hline
 \end{tabular}
\end{table}


Test whether the variances of the two sets of weight gains are significantly different. $Let \alpha = 0.05$. For the mice in the magnetic field, $s_X = 5.67$; for the other mice, $s_Y = 3.18$.

\subsection{Answer:}
$$\Big(\bar{x} - \bar{y} - t_{.025.m+n-2}*s_p*\sqrt{\frac{1}{n}+\frac{1}{m}} , \bar{x} - \bar{y} - t_{.975.m+n-2}*s_p*\sqrt{\frac{1}{n}+\frac{1}{m}}\Big)$$
$$\leftrightarrow$$
$$\Big(83.86-88.84  - 2.281\sqrt{\frac{1}{5}+\frac{1}{7}} , 83.86-88.84  + 2.281\sqrt{\frac{1}{5}+\frac{1}{7}}\Big)$$
$$\leftrightarrow$$
$$(-15.49, 13.73)$$
implies,
$$(0, 13.73)$$

\section{9.5.6}
Construct a 95\% confidence interval for the $\frac{\sigma_{x}^2}{\sigma^2_{y}}$ based on the data in Case Study 9.2.1. The hypothesis test referred to tacitly assumed that the variances were equal. Does that agree with your confidence interval? Explain.


\subsection*{Answer:}

$$(F_{1-.025,m-1,n-1}\frac{s_x^2}{s_y^2},F_{025,m-1,n-1}\frac{s_x^2}{s_y^2})$$
$$(.2383*(\frac{.0002103}{.0000955}),4,8232*(\frac{.0002103}{.0000955})$$
$$C.I. =(0.525,10.621)$$


\section{9.5.10}
Construct an 80\% confidence interval for the difference $p_m-p_w$ in the nightmare frequency data summarized in Case Study 9.4.2.

\subsection*{Answer:}
$$CI= (\bar{x}-\bar{y}- z_{\frac{\alpha}{2}}s_x, \bar{x}-\bar{y} + z_{\frac{\alpha}{2}}s_x)$$
$$CI= (0.031-.0645, 0.031+.0645)$$
$$CI= (-0.0335, .0955)$$

\section{10.2.6}
Based on his performance so far this season, a baseball player has the following probabilities associated with each official at-bat:

\begin{table}[h!]
\centering
 \begin{tabular}{c c } 
 \hline
Outcome & Probability\\
 \hline
Out & .713\\
Single & .270\\
Double & .010\\
Triple & .002\\
Home Run & .005\\
 \hline
 \end{tabular}
\end{table}

\subsection*{Answer:}
$$P(X_1=2,X_2=2,X_3+1,X_4=0,X_5=0)=\frac{5!}{2!2!1!0!0!}.270^2*.713^2.010^1*1*1=.011$$

\section{10.3.6}
A number of reports in the medical literature suggest that the season of birth and the incidence of schizophrenia may be related, with a higher proportion of schizophrenics being born during the early months of the year. A study (78) following up on this hypothesis looked at 5139 persons born in England or Wales during the years 1921?1955 who were admitted to a psychiatric ward with a diagnosis of schizophrenia. Of these 5139, 1383 were born in the first quarter of the year. Based on census figures in the two countries, the expected number of persons, out of a random 5139, who would be born in the first quarter is 1292.1. Do an appropriate $\chi^2$ test with ? = 0.05.

\subsection*{Answer:}
\textbf{H$_0:$} Incidence of schizophrenia is uniformly distributed across the year.
\newline
\textbf{H$_1:$} Incidence of schizophrenia is not uniformly distributed.
\newline
We have 1 degree of freedom. 
Therefore the $\chi^2_{.025} = 3.841$

Calculating the test statistic.
$$\hat{p}=np=1292.1$$
$$r=\sum\frac{(d_i-n\hat{p}^2)}{n\hat{p}}=\frac{(1383-1292.1)^2}{1292.1}=6.39$$

Since 6.39>3.841 we reject $h_0$. The incidence of admission for the disease is not uniformly distributed.

\section{10.3.9}
Records kept at an eastern racetrack showed the following distribution of winners as a function of their starting-post position. All one hundred forty-four races were run with a field of eight horses. 
\begin{table}[h!]
\centering
 \begin{tabular}{c c c c c c c c c} 

\textbf{Starting Post}      \vline & 1 & 2 & 3 & 4 & 5 & 6  & 7 & 8\\
 \hline
 \textbf{Number of Winners} \vline & 32 & 21 &19 & 20 & 16 & 11 &14 &11 \\
 \end{tabular}
\end{table}

Test the goodness-of-fit hypothesis that horses are equally likely to win from any starting position. Let $\alpha$ = 0.05.

\subsection*{Answer:}
\textbf{H$_0:$} The data fits the uniform distribution across starting places.
\newline
\textbf{H$_1:$} The data does not fit the distribution across places.
\newline
Our critical value is $\chi_{.95,7}=14.06$
$E[x] = \frac{144}{8}$
$$\sum \frac{(x_i - E[x])^2}{E[x]} = \frac{(32-13)^2}{18} + \frac{(21-13)^2}{18}+ \frac{(19-13)^2}{18}+ \frac{(20-13)^2}{18}+ \frac{(16-13)^2}{18}$$
$$+ \frac{(11-13)^2}{18}+ \frac{(14-13)^2}{18}+\frac{(11-13)^2}{18}=18.22$$

$18.22>14.06$. Therefore the data does not fit the distribution well and reject $H_0$. 

\section{10.4.10}
 In American football a turnover is defined as a fumble or an intercepted pass. The table below gives the number of turnovers committed by the home team in four hundred forty games. Test that these data fit a Poisson distribution at the 0.05 level of significance.

\begin{table}[h!]
\centering
 \begin{tabular}{c c} 
\hline
\textbf{Turnovers} & \textbf{Numbers Observed}\\
\hline
0 & 75 \\
1 & 125 \\
2 & 126 \\
3 & 60 \\
4 & 34 \\
5 & 13 \\
6+ & 7 \\
 \end{tabular}
\end{table}

\subsection*{Answer:}
\textbf{H$_0:$} The data fits the poisson distribution.
\newline
\textbf{H$_1:$} The data does not fit the poisson distribution.
\newline

We need to estimate $\lambda$
$$\hat{\lambda} = \sum^{6}_{1} f_i*n_i=\frac{800}{440}=1.81$$
Our degrees of freedom $df=n-k-l=6-1-1=4$
Our critical value $\chi^2_{4}=9.4877$

Now our test statistic:
$$\sum\frac{(x-\lambda)^2}{\lambda}=3.52$$

Since 3.52<9.48, we fail to reject $H_0$. The data fits the poisson distribution well. 


\section{10.5.6}
High blood pressure is known to be one of the major contributors to coronary heart disease. A study was done to see whether or not there is a significant relationship between the blood pressures of children and those of their fathers (96). If such a relationship did exist, it might be possible to use one group to screen for high-risk individuals in the other group. The subjects were ninety-two eleventh graders, forty-seven males and forty-five females, and their fathers. Blood pressures for both the children and the fathers were categorized as belonging to either the lower, middle, or upper third of their respective distributions. Test whether or not the blood pressures of children can be considered to be independent of the blood pressures of their fathers. Let $\alpha$ = 0.05.

\subsection*{Answer:}
\textbf{H$_0:$} Blood Pressure of child and father are independent. 
\newline
\textbf{H$_1:$} Blood Pressure of child and father are not independent.
\newline

Degrees of freedom, $df=(3-1)*(3-1)=4$
Critical Value $\chi^2_{.95,4}=9.488$

$$k=\sum\frac{(o-e)^2}{e}= \frac{(x_{ij}-n\hat{p}\hat{q})^2}{n\hat{p}\hat{q}}= .7461+.0199+$$
$$.5547+.0294+.0044+.0609+1.2504+.0070+1.1414 = 3.81$$

Since $3.81<9.48$ we fail to reject $H_o$ and the blood pressure of the child and parent for the assumption that they are independent well.

\section{R PROGRAMMING ASSIGNENT}
We have 1 occurrence of the sample means being less than -.33.

Therefore our p value is $p=\frac{1}{1000}=.0001$
 
R Programming Assignment:
\begin{lstlisting}
relcomp <- function(a, b) {
  
  comp <- vector()
  
  for (i in a) {
    if (i %in% a && !(i %in% b)) {
      comp <- append(comp, i)
    }
  }
  
  return(comp)
}

arm<-c(2.923,3.385,3.154,3.363,3.226,3.283,3.427,3.437,3.746,3.438)
fxd<-c(3.525,3.625,3.383,3.625,3.661,3.791,3.941,3.781,3.660,3.773)
diff<-mean(arm)-mean(fxd)
combined <- c(arm, fxd)
diffavg <- c()
lt<-0
for (i in 1:1000){
  smp <- sample(1:20,10)
  nsmp <-relcomp(1:20, smp)
  set1 <- combined [smp]
  set2 <- combined [nsmp]
  diffavg = c(diffavg, mean(set1)-mean(set2) )
  if (mean(set1)-mean(set2) < diff){
    lt <-lt+1
  }
}
\end{lstlisting}
\end{document}
